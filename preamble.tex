
\usepackage[T1]{fontenc}
\usepackage[utf8]{inputenc}
\usepackage{float}
\usepackage{graphicx}

%%%Todo lo de arriba viene de lyx y no se que hace

\usepackage{subfiles}						%MULTIARCHIVOS, NO BORRAR

%donde voy a buscar los archivos de imagenes 
\graphicspath{{Ej1/Informe/imagenes/}{Ej1/Informe/}
			  {Ej2/Informe/imagenes/}{Ej2/Informe/}
			  {Ej3/Informe/imagenes/}{Ej3/Informe/}
			  {Ej4/Informe/imagenes/}{Ej4/Informe/}
			  {Ej5/Informe/imagenes/}{Ej5/Informe/}
			  {Ej6/Informe/imagenes/}{Ej6/Informe/}}	
										

\usepackage{nomencl}					%Para introducir nomenclaturas/definciones
\usepackage{makecell}					%Para emprolijar celdas de tablas
\usepackage{amsmath}
\usepackage{amssymb}					%simbolos matematicos
\usepackage{upgreek}					%puedo usar \uptau que es como \tau pero con mas rulito
\usepackage{steinmetz}
\usepackage{mathtools}
\usepackage{placeins}
\usepackage[textwidth=16cm]{geometry}	%texto ocupa mas ancho de pagina
\usepackage{xcolor}						%se usa en \code
\usepackage[american]{circuitikz}		%dibujar esquematicos y diagramas
\usepackage[parfill]{parskip}			%pone espacio entre parrafos
\setlength{\parindent}{10pt}			%cuanta sangria al principio de un parrafo
\usepackage{indentfirst}				%pone sangria al primer parrafo de una seccion
\usepackage{gensymb}
\usepackage{textcomp}
\usepackage[hidelinks]{hyperref}
\usepackage{pdfpages}
\usepackage{csvsimple}

% Swap the definition of \abs* and \norm*, so that \abs
% and \norm resizes the size of the brackets, and the 
% starred version does not.
\DeclarePairedDelimiter\abs{\lvert}{\rvert} %
\makeatletter	%magia de categoria de caracteres en Tex, ignorar
\let\oldabs\abs 
\def\abs{\@ifstar{\oldabs}{\oldabs*}}
\let\oldnorm\norm
\def\norm{\@ifstar{\oldnorm}{\oldnorm*}}
\makeatother	%magia de categoria de caracteres en Tex, ignorar

%Definicion comando \parsum: hace re piola el simbolo de la suma en paralelo
\newcommand{\parsum}{\mathbin{\!/\mkern-5mu/\!}} 

%Definicion comando \code: poen el texto en fuente monoespaciada con fondo gris 
%al estilo del codigo de stack overflow
\definecolor{light-gray}{gray}{0.95} 
\newcommand{\code}[1]{\colorbox{light-gray}{\texttt{#1}}}

%TODOs
\usepackage[colorinlistoftodos]{todonotes}

%
\usepackage{subcaption}


%cambia las palabras que estan en ingles a castellano
\AtBeginDocument{\renewcommand\contentsname{Índice}}
\AtBeginDocument{\renewcommand\figurename{Figura}}


\makeatletter