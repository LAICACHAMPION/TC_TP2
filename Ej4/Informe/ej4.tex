\documentclass[../../main.tex]{subfiles}

\begin{document}

\section{Circuitos integradores y derivadores}

Algunas aplicaciones \'utiles de circuitos con amplificadores operacionales implican realizar operaciones matem\'aticas entre las se\~nales involucradas en un circuito. En esta secci\'on estudiaremos los casos particulares de derivaci\'on e integraci\'on con \textit{op amps}. En ambos circuitos, se utilizar\'a el operacional \textit{LM833}, as\'i como una resistencia de $R = 15k\Omega$ y un capacitor de $C = 6.8nF$.

\subsection{An\'alisis matem\'atico} \label{ssection:formulas}

\begin{figure}[htb]
	\centering
	\begin{circuitikz}
  		\draw (0,0) node[op amp] (opamp) {}
  		(opamp.-) to [generic, l_=$Z_1$, *-o] ($(opamp.-)-(2,0)$) node[left]{$V_{in}$}
  		(opamp.-) |- ($(opamp.-)+(0.2,1)$) to[generic, l=$Z_2$] ($(opamp.-)+(2.2,1)$) -|
  		(opamp.out) to[short,*-] ($(opamp.out)+(.5,0)$) node [right] {$V_{out}$} node [ocirc] {} 
  		(opamp.+) to[short] ($(opamp.+) - (0,.5)$) node[ground] {}
  ;
\end{circuitikz}
	\caption{Circuito inversor}
\end{figure}

Dado que todos los circuitos estudiados en esta secci\'on presentan la misma topolog\'ia general, analizaremos el caso general para cada modelo de operacional, y para obtener los resultados particulares bastar\'a reemplazar en el resultado final con los valores de $Z_1$ y $Z_2$ que corresponda.

\subsubsection{$A_0$ infinito}
Si consideramos que $V^-=V^+$, entonces este circuito presenta una tierra virtual en ese punto, y por lo tanto puede resolverse trivialmente, obtieniendo:

\begin{equation} \label{eq:tf-ideal} H(s) = -\frac{Z_2(s)}{Z_1(s)} \end{equation}
\begin{equation} \label{eq:zin-ideal} Z_{in}(s) = Z_1(s) \end{equation}


\subsubsection{$A_0$ finito}
Al considerar que la ganancia no es infinita, ya no se cumple que $V^-=0$, aunque mientras que sigamos admitiendo que la impedancia del operacional es infnita, existe una sola corriente en el circuito y por lo tanto las ecuaciones quedan planteadas como:'

 \[
	\left\{
 	\begin{array}{ll}
		V_{in} - V_{out} = I\cdot (Z_1+Z_2)\\
		V^- = V_{out} + I\cdot Z_2\\
		V_{out} = - A_0 \cdot V^-
	\end{array}
	\right.
 \]

En este caso, el resultado obtenido es:
\begin{equation} \label{eq:tf-ao} H(s) =-\frac{A_0\cdot Z_2}{Z_2+(A_0+1)\cdot Z_1} 
						\sim-\frac{A_0\cdot Z_2}{Z_2+A_0\cdot Z_1}  \end{equation}
\begin{equation} \label{eq:zin-ao} Z_{in}(s) = \frac{Z_2}{A_0+1} +Z_1 \sim  \frac{Z_2}{A_0} +Z_1 \end{equation}

Podemos verificar la validez de estas expresiones notando que $\lim_{A_0\to\infty}$ llegamos, en ambos casos, a los resultados de la secci\'on anterior.\par

Para el operacional utilizado, el valor de $A_0$ es $110dB$.

\subsubsection{$A_{vol}(s)$}
Para obtener la f\'ormula del modelo de polo dominante aplicado a este circuito, basta reemplazar $A_0$ por $A_{vol}(s)=\frac{A_0}{\frac{s}{\omega_p} +1}$ en las ecuaciones \ref{eq:tf-ao} y \ref{eq:zin-ao}. Se obtiene entonces:

\begin{equation} \label{eq:tf-avol} H(s) =-\frac{A_0\cdot Z_2}{\left(\frac{s}{\omega_p}+1\right)\left(Z_2+Z_1\right) + A_0 \cdot Z_1}  \sim
- \left(\frac{A_0 \cdot Z_2}{A_0 \cdot Z_1 + Z_2}\right) \cdot \left(\frac{1}{\left(\frac{Z_1 + Z_2}{A_0 \cdot Z_1 + Z_2} \cdot \frac{1}{\omega_p}\right) \cdot s + 1}\right)
\end{equation}
\begin{equation} \label{eq:zin-avol} Z_{in}(s) = \frac{\left(\frac{s}{\omega_p}+1\right)\left(Z_2+Z_1\right)+ A_0 \cdot Z_1}{\frac{s}{\omega_p}+A_0+1}
\sim \left(\frac{Z_2 + A_0 \cdot Z_1}{A_0}\right) \cdot \left(\frac{ \left(\frac{Z_1+Z_2}{Z_2+A_0\cdot Z_1}\right)\cdot \frac{1}{\omega_p} \cdot s + 1 }{ \frac{1} {A_0 \cdot \omega_p} \cdot s + 1}\right) \end{equation}

En este caso tambi\'en se verifica que el t\'ermino que no depende de $\omega_p$ tiende a la ganancia cuando $A_0$ tiende a infinito. Sin embargo, se agrega un polo a la transferencia, y un polo y un cero a la impedancia.\par

En el \textit{LM833}, dado que el valor del $BWP = 16MHz$, $\omega_p = 2\pi \frac{BWP}{A_0} \sim 2\pi \cdot 50.6 Hz$.










\subsection{Derivador}

Para armar un circuito derivador con los componentes mencionados, la conexi\'on debe realizarse de la siguiente manera:

\begin{figure}[htb]
	\centering
	\begin{circuitikz}
  		\draw (0,0) node[op amp] (opamp) {}
  		(opamp.-) to [C, l_=$C$, *-o] ($(opamp.-)-(2,0)$) node[left]{$V_{in}$}
  		(opamp.-) |- ($(opamp.-)+(0.2,1)$) to[R=$R$] ($(opamp.-)+(2.2,1)$) -|
  		(opamp.out) to[short,*-] ($(opamp.out)+(.5,0)$) node [right] {$V_{out}$} node [ocirc] {} 
  		(opamp.+) to[short] ($(opamp.+) - (0,.5)$) node[ground] {}
  ;
\end{circuitikz}
	\caption{Circuito derivador}
\end{figure}


\subsubsection{An\'alisis matem\'atico: respuesta en frecuencia}

Si consideramos el modelo ideal para el \textit{op amp}, al tener una tierra virtual en $V^-$, la entrada y la salida est\'an aisladas entre s\'i, reemplazando en \ref{eq':tf-ideal} $Z_1=\frac{1}{sC}$ y $Z_2=R$:

\[ H(s) = -RC \cdot s \]

Antitransformando esta expresi\'on, obtenemos que $v_{out}(t) = -RC \cdot \frac{\partial}{\partial t}v_{in}(t)$, con lo cual anal\'iticamente podemos ver que cumple la funci\'on planteada inicialmente, si bien la salida estar\'a invertida y multiplicada por una constante.  \par

Con el modelo de $A_0$ constante, en cambio, la ecuaci\'on resultante es:

\[ H(s) = -\left(\frac{A_0 \cdot RC}{1+A_0}\right) \cdot \left(\frac{s}{ \left(\frac{RC}{A_0+1}\right) \cdot s + 1\ }\right)\]

Dado que $A_0+1\sim A_0$, la constante es pr\'acticamente id\'entica a la del modelo ideal, pero en este caso se agrega a la transferencia un polo en $f= \frac{A_0+1}{2\pi \cdot RC} \sim 493MHz$. Por lo tanto, sus efectos no ser\'ian apreciables hasta llegar a frecuencias en el orden de los $MHz$, con lo cual hasta frecuencias de $kHz$ el circuito deber\'ia derivar correctamente.

Por \'ultimo, teniendo en cuenta el polo dominante del operacional, la funci\'on transferencia queda reducida a:

\[ H(s) = -\left(\frac{A_0 \cdot RC}{1+A_0}\right) \cdot 
	\left( \frac{s} { \left( \frac{RC}{(1+A_0)\cdot \omega_p} \right)\cdot s^2 + \left( \frac{RC\cdot \omega_p +1}{(1+A_0)\cdot\omega_p} \right)\cdot s + 1  } \right) \]
	
En este caso, el polo queda de segundo orden, con $f_0 = \frac{1}{2\pi} \sqrt{\frac{(1+A_0)\cdot \omega_p}{RC}} = 158kHz$, con $\xi = \frac{\omega_0 \cdot (RC\cdot\omega_p+1)} {2\omega_p\cdot(1+A_0)} = 0.005$. La respuesta en frecuencia, entonces, presentar\'a un sobrepico considerable en esta frecuencia, y un salto abrupto de $-180^\circ$ en la fase. Sin embargo, aqu\'i no se est\'an teniendo en cuenta los $50\Omega$ de impedancia del generador de funciones que quedar\'an en serie con el circuito, que provocar\'ian que el sobrepico no sea tan pronunciado. Esto se tratar\'a m\'as en detalle en la secci\'on \ref{ssection:dcomp}. \par

Como el \'ultimo modelo introduce un cambio tan grande en el comportamiento del circuito, ser\'a el que contrastaremos con los resultados. Se espera que el circuito derive se\~nales con frecuencia menor a la del polo. El alto factor de calidad sugerir\'ia que no se empezar\'ian a observar cambios hasta frecuencias del mismo orden que ella.



\subsubsection{An\'alisis matem\'atico: impedancia de entrada}

Idealmente, la impedancia de entrada del circuito ser\'ia solo la del capacitor. Si utiliz\'azemos la expresi\'on \ref{eq:zin-ao}, deber\'iamos adem\'as sumar $\frac{R}{A_0}\sim 0.05\Omega$, pero esto ser\'ia comparable con la impedancia del capacitor s\'olo en frecuencias del orden de los $100MHz$, con lo cual despreciaremos su aporte.\par
	
Con el modelo de $A_{vol}(s)$, la funci\'on que se obtiene es:

\[ Z_{in}(s) = \frac{1}{sC} \cdot \left( \frac{ \frac{RC}{(1+A_0)\cdot \omega_p} \cdot s^2 + \frac{1 + RC \cdot \omega_p}{(A_0+1) \cdot \omega_p} \cdot s + 1 }{\frac{1}{(1+A_0)\cdot\omega_p}\cdot s+1 } \right)\]	

Al igual que la transferencia, esta funci\'on tiene $f_0 = 158kHz$ con un $\xi = 0.05$, pero en este caso en un cero en lugar de un polo. El polo que s\'i presenta esta funci\'on est\'a en $16MHz$. 



\subsubsection{An\'alisis de resultados: respuesta en frecuencia} \label{ssection:d-hf}

\todo[inline]{BODES SUPERPUESTOS AAA}

En la figura anterior, se observa que el modelo logr\'o precedir correctamente la frecuencia del polo de segundo orden, as\'i como la presencia de un sobre pico. Sin embargo, no se pudo medir en frecuencias muy cercanas a este punto, debido a limitaciones del \textit{slew rate} de $7\frac{V}{\mu s}$ del operacional. A $158k\Omega$, si estimamos que la ganancia ser\'ia de $50dB$ como calcula el simulador, la m\'axima tensi\'on de entrada admisible ser\'ia de $V_{in} = \frac{7\frac{V}{\mu s}}{2\pi \cdot 150kHz \cdot 10^{50/20}}\sim 22mV$, lo cual es del orden del ruido del osciloscopio y por lo tanto no ser\'ia una medici\'on confiable.\par

Observando el comportamiento de la fase, podemos estimar que la predicci\'on del simulador es mejor que la anal\'itica. El hecho de que la fase contin\'ua decreciendo m\'as all\'a de los $90^\circ$ sugerir\'ia que hay otra singularidad en el sistema, que proviene de alg\'un par\'ametro del operacional que el simulador tiene en cuenta y nosotros no. Puesto que la \textit{data sheet} informa que la frecuencia en la cual el operacional tiene ganancia unitaria es $9MHz$, en lugar de los 16 que indicar\'ia el \textit{bandwidth product}, es razonable suponer que el operacional tiene otro polo de frecuencia mucho mayor a la del primero, que llega a apreciarse debido a que se est\'a trabajando a frecuencia y ganancia elevadas.\par 

Otra informaci\'on que se puede extraer de la fase es el rango de frecuencias donde el circuito deriva. La fase se mantuvo en el rango $(-90\pm 3)^\circ$ hasta $f = 20kHz$. M\'as all\'a de ese punto, se considera que no se puede utilizar el circuito como derivador. 




\subsubsection{An\'alisis de resultados: impedancia de entrada}

Las mediciones presentadas a continuaci\'on se realizaron colocando una resistencia de $10k\Omega$ en serie con el circuito, y asumiendo que la misma no introduce cambios de fase en el rango de frecuencias donde se trabaj\'o. \par

\todo[inline]{BODEEEEEEEEEEEEEEEEEEEEEEEEE}

Para el rango de frecuencias medido, el comportamiento es pr\'acticamente id\'entico al ideal: la fase se mantiene constante en $-90^\circ$, y la magnitud baja $20dB$ por d\'ecada. No se pudieron hacer mediciones m\'as all\'a de los $120kHz$ debido a las limitaciones explicadas en la secci\'on anterior. En la \'ultima medici\'on se llega a apreciar que el descenso en magnitud es m\'as abrupto, lo cual coincidir\'ia con la presencia del cero de orden dos que se observa en la teor\'ia y en Spice. \par

Si asumimos que la impedancia medida es puramente capacitiva, podemos calcular para cada medici\'on $C = (2\pi \cdot f \cdot 10^{\abs{H}/20})^{-1}$. Salvo para el \'utimo punto, se obtienen valores de $C$ entre 6.9 y $9nF$. Siendo que el valor del capacitor utilizado era $6.8nF\pm 5\%$,  estos valores indicar\'ian que una parte de la impedancia proviene de otros elementos, pero de todas formas el orden de magnitud es el adecuado. Para la \'ultima medici\'on, sin embargo, se obtiene $C=12.4pF$. Esto refuerza la idea de que en esta medici\'on influye el cero de segundo orden proveniente del polo del operacional.

\subsubsection{An\'alisis de resultados: respuesta transitoria}

Seg\'un lo medido en la secci\'on \ref{ssection:d-hf}, deber\'ian poder derivarse se\~nales de $f\leq 20kHz$. 

\todo[inline]{FOTO 1K DERIVADOR}

Aqu\'i se observa que el circuito deriva correctamente la se\~nal de entrada. Cabe aclarar que la salida se muestra invertida para que se aprecie el efecto derivador, pues como ya se mencion\'o la salida est\'a multiplicada por $(-1)$. \par

Algo que llama la antenci\'on en esta foto son los picos cuando la pendiente de la entrada cambia de signo.

\todo[inline]{FOTO TRANSITORIO 1K DERIVADOR}

Se observa que el circuito oscila antes de estabilizarse. Esto es consistente con el hecho de que los polos del sistema son complejos conjugados, es decir, con que el sistema es subamortiguado.\par

\todo[inline]{FOTO TRANSITORIO 50K DERIVADOR}

Cuando observamos la respuesta de una frecuencia donde la fase ya no es cercana a $-90^\circ$, la salida no coincide con la derivada de la entrada. Esto tambi\'en puede explicarse con que, como se observa en la imagen, el circuito no llega a estabilizarse en un per\'iodo y s\'olo se observa la respuesta transitoria. 



\subsection{Derivador compensado} \label{ssection:dcomp}
En la secci\'on anterior, la elevada ganancia del sistema en la frecuencia del polo impidi\'o que se pudiesen tomar mediciones en un gran rango de frecuencia. Por lo tanto, procederemos a continuaci\'on a compensar este comportamiento.\par

Si observamos la funci\'on transferencia ideal del circuito derivador, observamos que la ganancia se hace infinita cuando la frecuencia tambi\'en tiende a infinito. Esto se debe a que el sistema presenta un cero en el origen, que proviene de que para frecuencias altas la impedancia del capacitor disminuye y tiende a cero. Esto puede solucionarse imponiendo una impedancia m\'inima independiente de la frecuencia, lo cual se puede lograr colocando un resistor en serie con el capacitor.\par 

\begin{figure}[htb]
	\centering
	\begin{circuitikz}
  		\draw (0,0) node[op amp] (opamp) {}
  		(opamp.-) to [C, l_=$C$, *-o] ($(opamp.-)-(2,0)$) 
		to [R, l_=$R_C$, *-o]  ($(opamp.-)-(4,0)$) node[left]{$V_{in}$}
  		(opamp.-) |- ($(opamp.-)+(0.2,1)$) to[R=$R$] ($(opamp.-)+(2.2,1)$) -|
  		(opamp.out) to[short,*-] ($(opamp.out)+(.5,0)$) node [right] {$V_{out}$} node [ocirc] {} 
  		(opamp.+) to[short] ($(opamp.+) - (0,.5)$) node[ground] {}
  ;
\end{circuitikz}
	\caption{Circuito derivador compensado}
\end{figure}

\subsubsection{An\'alisis matem\'atico: respuesta en frecuencia}
El an\'alisis de este circuito es equivalente al efectuado en la secci\'on \ref{ssection:formulas}, efectuando las sustituciones  $Z_1=R_C+\frac{1}{sC}$ y $Z_2 = R$. 

La funci\'on transferencia que se obtiene es:

\begin{equation} H(s) = -\left(\frac{A_0 \cdot RC \cdot s}{1+A_0}\right) \cdot
\left(\frac{1}{ \frac{(R+R_C)\cdot C}{(1+A_0)\cdot \omega_p} \cdot s^2 + [\frac{1}{\omega_p}+ C\cdot(R+R_C\cdot(1+A_0))]\cdot \frac{1}{1+A_0} \cdot s + 1 }\right)\end{equation}

Por lo tanto, para eliminar el sobrepico debemos obtener el valor de $R_C$ tal que $xi\geq 0.707$. Esto se resolvi\'o con el siguiente c\'odigo en \textit{Matlab}:\\
\code{r = 15e3; c = 6.8e-9;}\newline
\code{Ao = 10\^(110/20); \% de la hoja de datos: Ao=110dB}\\
\code{BWP = 16e6;}\\
\code{wp = 2*pi*BWP/Ao;}\\
\\
\code{syms r2;}\\
\code{w0 = sqrt(wp*(1+Ao)/c/(r+r2));}\\
\code{xi = w0/2*(c*(r+r2*(1+Ao))+1/wp)/(1+Ao);}\\
\code{r2 = eval(solve(xi == 0.707, r2));}

Se obtiene as\'i que $R_C \geq 210\Omega$. Sin embargo, si se tomase la m\'inima indispensable para quitar el sobrepico, se tendr\'ia una ganancia de aproximadamente $40dB$ en la frecuencia del polo. Por lo tanto, se utiliz\'o $R_C = 470\Omega$, con la cual la magnitud no deber\'ia superar los $30dB$. \par

Como ahora el sistema esta sobreamortiguado, los polos ya no son complejos conjugados sino dos polos reales distintos. Calculando $\omega_0$ y $\xi$ con el valor elegido de $R_2$, las frecuencias de corte que se obtienen son 

\subsubsection{An\'alisis matem\'atico: impedancia de entada}
En este caso, la impedancia de entrada ideal es $R_C$ en serie con el capacitor:

\[ H(s) = \frac{1}{C} \cdot \frac{R_C C \cdot s + 1}{s} \]
 
 Esta transferencia cuenta con un polo en el origen y un cero en $f = \frac{1}{2\pi \cdot R_C C} \sim 50kHz$.\par

De igual manera que el caso anterior, se desprecian los $0.05\Omega$ que se suman en el modelo de $A_0$ constante.\par

Considerando, en cambio, el modelo de polo dominante, la impedancia de entrada que se obtiene es:

\begin{equation}Z{in}(s) = \frac{1} {s \cdot C} \cdot \left(\frac{ \frac{(R+R_C)\cdot C}{\omega_p}\cdot s^2 +  [\frac{1}{\omega_p}+ C\cdot(R+R_C\cdot(1+A_0))]\cdot \frac{1}{1+A_0}\cdot s + 1 }{ \frac{1}{(1+A_0)\cdot \omega_p}\cdot s + 1 }\right)\end{equation}

Los ceros de esta funci\'on est\'an en los polos de la transferencia, es decir que tiene un , y se agrega adem\'as un polo en $f_1 =  56kHz$ y $f_2 = 432kHz$. Teniendo en cuenta que se realizar\'an mediciones en ese rango, se utilizar\'a este modelo para comparar con las mediciones.

\subsubsection{An\'alisis de resultados: respuesta en frecuencia}

En este caso s\'i se pudieron efectuar mediciones en un rango continuo de mediciones gracias a la ausencia del sobrepico. Los resultados obtenidos fueron:

\todo[inline]{BODES RTA FREC DCOMP}

El modelo predice adecuadamente el comportamiento del circuito. El hecho de que la magnitud m\'axima medida no coincida con la calculada ni la simulada podr\'ia atribuirse a que las mediciones se realizaron con tasas de cambio de $V_{out}$ cercanas, si bien inferiores, al \textit{slew rate} del operacional. Este efecto se habr\'ia visto exacerbado de haber elegido una resistencia de compensaci\'on menor.\par

En base a los resultados obtenidos para la fase, el nuevo circuito integra hasta $f = 2k\Omega$, es decir un orden de magnitud menos que el derivador no compensado.


\subsubsection{An\'alisis de resutlados: impedancia de entrada}

\todo[inline]{EL BODE ZIN DCOMP}

Las mediciones cumplen las predicciones del simulador y las anal\'iticas, que esta vez coinciden entre ellas. Se logr\'o satisfactoriamente lograr limitar el m\'inimo de impedancia de entrada.


\subsubsection{An\'alisis de resultados: respuesta transitoria}

Repetiremos la medici\'on que realizamos para el circuito no compensado en $f=1kHz$.

\todo[inline]{FOTO 1K DCOMP}

El sistema conserva su comportamiento de derivador para frecuencias menores a $2kHz$ como se esperaba. En este caso, adem\'as, ya no se produce un \textit{overshoot} en el transitorio.

\todo[inline]{FOTO 1K DCOMP TRANSITORIO}

Ahora el transitorio corresponde al de un circuito de segundo orden sobreamortiguado, que es el comportamiento que dese\'abamos que presentase el circuito compensado.


\todo[inline]{FOTO 100K DCOMP}
Para frecuencias altas, sin embargo, el circuito ya no se comporta como un integrador. Esto sucede porque el per\'iodo de la se\~nal es menor que el tiempo del transitorio del circuito.





\subsection{Integrador}

Colocando los componentes en el orden inverso al derivador se obtiene el circuito integrador:


\begin{figure}[htb]
	\centering
	\begin{circuitikz}
  		\draw (0,0) node[op amp] (opamp) {}
  		(opamp.-) to [R, l_=$R$, *-o] ($(opamp.-)-(2,0)$) node[left]{$V_{in}$}
  		(opamp.-) |- ($(opamp.-)+(0.2,1)$) to[C=$C$] ($(opamp.-)+(2.2,1)$) -|
  		(opamp.out) to[short,*-] ($(opamp.out)+(.5,0)$) node [right] {$V_{out}$} node [ocirc] {} 
  		(opamp.+) to[short] ($(opamp.+) - (0,.5)$) node[ground] {}
  ;
\end{circuitikz}
	\caption{Circuito integrador}
\end{figure}

\subsubsection{An\'alisis matem\'atico: respuesta en frecuencia}

 Reemplazando $Z_1$ por $R$ y $Z_2$ por $\frac{1}{sC}$ en la ecuaci\'on \ref{eq:tf-ideal}, obtenemos:


\[ H(s) =  -\frac{1}{RC\cdot s}\]

Efectuando la antitransformada de Laplace a esta expresi\'on, resulta que $v_{out}(t) = -\frac{1}{RC} \int_{-\infty}^t v(u)du$, es decir que a la salida se obtiene la integral de la se\~nal de la entrada, invertida y multiplicada por la constante \frac{1}{RC}. As\'i planteado, este sistema tiene ganancia infinita para corriente continua. Esto podr\'ia ser un problema debido a que cualquier ruido de frecuencias bajas se ver\'a enormemente amplificado. \par

Utilizando la ecuaci\'on \ref{eq:tf-ao}, la nueva transferencia que obtenemos es:


\[ H(s) = -  \frac {A_0}{(A_0+1)\cdot RC \cdot s+1} \]

Aqu\'i el polo se traslada del origen a $f = \frac{1}{2\pi \cdot RC \cdot (A_0+1)} \sim 5mHz$. Esto establece un m\'aximo de ganancia para continua, pero en una frecuencia tan baja que ser\'ia razonable esperar que sea un problema de todas maneras. \par

Por \'ultimo, considerando el polo del operacional la transferencia final queda en:

\begin{equation}\label{eq:tf-int} H(s) = -\frac{A_0}{\frac{RC}{\omega_p} \cdot s^2 + \left(\frac{1}{\omega_p} + (A_0+1) \cdot RC\right) \cdot s +1} \end{equation}

Esta funci\'on tiene tambi\'en un polo en $0.005Hz$, pero tiene un segundo en $16MHz$. Sin embargo, para esta frecuencia la atenuaci\'on probablemente sea tal que no se pueda medir la salida, puesto que los generadores de funciones utilizados para medir s\'olo pueden entregar hasta $20V_{pp}$. Por lo tanto, no podr\'ia en principio apreciarse un polo en esa frecuencia.


\subsubsection{An\'alisis matem\'atico: impedancia de entrada}

Idealmente, debido a la tierra virtual en $V^-$, la entrada s\'olo se carga con $Z_1 = R$, con lo cual la transferencia ser\'ia constante:

\[ Z_{in}(s) =  R\]

Si consideramos la expresi\'on \ref{zin-ao}, obtenemos en cambio que:

\[ Z_{in}(s) = \frac{(A_0+1)\cdot RC \cdot s +1}{ (A_0 +1) \cdot C \cdot s }\]

Seg\'un esta expresi\'on, tendr\'iamos un polo en el origen y un cero en $0.005Hz$. La ganancia que tendr\'ia el sistema en esas frecuencias ser\'ia, sin embargo, tan elevada que impidir\'ia medir la entrada sin que la salida sature, con lo cual s\'olo se podr\'ia medir en frecuencias donde los efectos del polo y el cero ya fueron neutralizados entre s\'i, y la expresi\'on se ver\'ia nuevamente reducida a $Z_{in}=R$.\par

Finalmente, con el modelo de $A_{vol}$ la impedancia de entrada resulta ser:

\begin{equation}Z_{in}(s) = \frac{\frac{RC}{\omega_p} \cdot s^2 + \left(\frac{1}{\omega_p} + (A_0+1) \cdot RC\right) \cdot s +1}{sC \cdot (A_0+1) \cdot \left(\frac{1}{\omega_p (A_0+1)}\cdot s +1\right) } \end{equation}

Los polos de la transferencia mencionados en \ref{eq:tf-int} son ahora ceros de la impedancia. Esta funci\'on cuenta tambi\'en con un polo en $f = \frac{omega_p(A_0+1)}{2\pi} \sim BWP = 16MHz$ , que es la misma frecuencia de uno de los ceros. Esto 

\todo[inline]{EN RESULTADOS DECIR QUE HABIA QUE TENER MUCHO CUIDADO CON EL OFFSET PORQUE TE MONTABA LA SALIDA SOBRE UNA CONTINUA ENORME Y SATURABA AL TOK}



\subsection{Integrador compensado}


\begin{figure}[htb]
	\centering
	\begin{circuitikz}
	
  		\draw (0,0) node[op amp] (opamp) {}
  		(opamp.-) to [R, l_=$R$, *-o] ($(opamp.-)-(2,0)$) node[left]{$V_{in}$}
  		(opamp.-) |- ($(opamp.-)+(0.2,1)$) to[C=$C$] ($(opamp.-)+(2.2,1)$) -|
  		(opamp.out) to[short,*-] ($(opamp.out)+(.5,0)$) node [right] {$V_{out}$} node [ocirc] {} 
  
  		 ($(opamp.-)+(0.2,1)$)  
  		 to [short, *-] ($(opamp.-)+(0.2,3)$) 
		to [R,  l_=$R_C$, *-] ($(opamp.-)+(2.2,3)$) 
 		to  [short, *-]($(opamp.-)+(2.2,1)$)
 		
 		(opamp.+) to[short] ($(opamp.+) - (0,.5)$) node[ground] {}
  ;
\end{circuitikz}
	\caption{Circuito integrador compensado}
	\end{figure}



\end{document}
